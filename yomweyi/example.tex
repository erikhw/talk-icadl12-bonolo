% Copyright 2009 by Tomasz Mazur
%
% This file may be distributed and/or modified in all ways.


\documentclass[xcolor=pdftex,t,11pt]{beamer}

%%%%%%%%%%%%%%%%%%%%%%%%%%%%%%%%%%
%       SET OPTIONS BELOW        %
%%%%%%%%%%%%%%%%%%%%%%%%%%%%%%%%%%

\usetheme[
% Toggle showing page counter
pagecounter=true,
%
% String to be used between the current page and the
% total page count, e.g. of, /, from, etc.
pageofpages=of,
%
% Defines the shape of bullet points. Available options: circle, square
bullet=circle,
%
% Show a line below the frame title. 
titleline=true,
%
% Set the style of the title page (true for fancy, false for standard)
alternativetitlepage=true,
%
% Institution logo for fancy title page.
% Comment out to remove the logo from the title page.
% IMPORTANT: THERE IS A BUG IN SOME VERSIONS OF PDFLATEX AND FONTS
% ON THE LOGOS ARE NOT RENDERED PROPERLY. IN SUCH A CASE ADD `2` 
% TO THE NAME OF THE LOGO, E.G. comlab2 INSTEAD OF comlab
titlepagelogo=images/titlepage/ou,
%
% Department footer logo for fancy title page
% Comment out to remove the logo from the footer of the title page/
% IMPORTANT: THERE IS A BUG IN SOME VERSIONS OF PDFLATEX AND FONTS
% ON THE LOGOS ARE NOT RENDERED PROPERLY. IN SUCH A CASE ADD `2` 
% TO THE NAME OF THE LOGO, E.G. comlab2 INSTEAD OF comlab
%titlepagefooterlogo=images/titlepage/comlab,
%
% Institution/department logo for ordinary slides
% Comment this line out to remove the logo from all the pages.
% Available logos are: ou, comlab, comlabinline, comlabou
% IMPORTANT: THERE IS A BUG IN SOME VERSIONS OF PDFLATEX AND FONTS
% ON THE LOGOS ARE NOT RENDERED PROPERLY. IN SUCH A CASE ADD `2` 
% TO THE NAME OF THE LOGO, E.G. comlab2 INSTEAD OF comlab
ordinarypageslogo=images/comlab,
%
%
% Add watermark in the bottom right corner
%watermark=<filename>,
%
% Set the height of the watermark.
%watermarkheight=100pt,
%
% The watermark image is 4 times bigger than watermarkheight.
%watermarkheightmult=4,
]{Torino}

% Select color theme. Available options are:
% mininmal, greenandblue, blue, red
\usecolortheme{blue}

%Select different font themes.Available options are:
% default, serif, structurebold, structureitalicserif, structuresmallcapsserif
\usefonttheme{structurebold}


%%%%%%%%%%%%%%%%%%%%%%%%%%%%%%%%%%
%       PRESENTATION INFO        %
%%%%%%%%%%%%%%%%%%%%%%%%%%%%%%%%%%

\author{Tomasz Mazur}
\title{Slideshow presentations using \LaTeX{}}
%\subtitle{}
\institute{Oxford University}
\date{\today}

\begin{document}



%%%%%%%%%%%%%%%%%%%%%%%%%%%%%%%%%%
%       SLIDE DEFINITIONS        %
%%%%%%%%%%%%%%%%%%%%%%%%%%%%%%%%%%

\begin{frame}[plain]
	\titlepage
\end{frame}

\begin{frame}{Outline}
	\tableofcontents
\end{frame}


\section{Introduction}


\subsection{What is this?}

\begin{frame}[fragile]
\frametitle{Introduction}
\begin{itemize}
\item \alert{Creating presentations using \LaTeX{} is straightforward...}
\item ...with Beamer, a class for creating slides
\item should be already installed with most \LaTeX{} distributions, but can be obtained from \verb!http://latex-beamer.sourceforge.net/!
\item Beamer documentation available from \verb!	http://www.ctan.org/tex-archive/macros/latex/! \verb!	contrib/beamer/doc/beameruserguide.pdf!
\item This is a modification of Marco Barision's Torino theme 
\item It aims to produce slides that are \emph{pretty}, but easily \emph{readable} and with \emph{large content area}
\item Most of standard Beamer commands are supported
\end{itemize}
\end{frame}

\subsection{Creating presentation overview}

\begin{frame}[fragile]
\frametitle{Creating your presentation}
\begin{itemize}
\item You can simply modify this file
\item Set the configuration options at the top of the document for
\begin{itemize}
	\item colours
	\item fonts
	\item title page style
	\item logos
	\item bullet points shapes, etc.
\end{itemize}
\item Compile using \verb!pdflatex! (recommended), but \verb!latex! works too
\item Due to format restrictions, graphics may be slightly misaligned in PS files, use PDF instead
\end{itemize}
\end{frame}


\section{Modifying themes, colours and fonts}


\begin{frame}[fragile]
\frametitle{Themes and colours}
\begin{itemize}
\item There are four basic colour themes: 
\begin{itemize}
	\item \verb!minimal! (least eye-candy, good for longer presentations)
	\item \verb!greenandblue! 
	\item \verb!blue! (good for low quality projectors)
	\item \verb!red! (American-style)
\end{itemize}
\item Themes define the colours of the background, slide decorations, slide titles, main text, bullet points, etc.
\item Edit themes by modifying the \verb!beamercolortheme*.sty! file
\end{itemize}
\end{frame}

\begin{frame}[fragile]
\frametitle{Fonts}
\begin{itemize}
\item There are five font themes: 
\begin{itemize}
	\item \verb!default! (sans serif)
	\item \verb!serif! (used for this presentation) 
	\item \verb!structurebold! (titles, headlines, etc.\ are typeset in a bold font)
	\item \verb!structureitalicserif! (titles, headlines, etc.\ are typeset in an italics serif font)
	\item \verb!structuresmallcapsserif! (titles, headlines, etc.\ are typeset in a small caps serif font)
\end{itemize}
\item Change the document-wise font size to 8, 9 , 10, 11 (default), 12, 14, 17 or 20 points in the options of \verb!\documentclass!, e.g. \verb!\documentclass[12pt]{beamer}!
\item Colour text using \verb!\textcolor{<<colour>>}{<<text>>}!
\item The \verb!\alert{<<text>>}! command colours text red
\end{itemize}
\end{frame}


\section{Adding things}


\subsection{Adding new slides}

\begin{frame}[fragile,allowframebreaks]
\frametitle{Adding slides}
\framesubtitle{...with subheadings}
\begin{itemize}
\item A slide is created using the following code:
	\begin{verbatim}
	 \begin{frame}[<<options>>]
	   \frametitle{<<slide title>>}
	   <<contents>>
	 \end{frame}
	\end{verbatim}
\item The possible options include:
\begin{itemize}
	\item \verb!plain! removes all slide decorations (useful for larger images)
	\item \verb!c! and \verb!b! align contents of the slide in the middle or bottom (default alignment is top, but this can easily be changed in the document class options)
	\item \verb!fragile! is necessary for slides that use the \verb!verbatim!
	\item \verb!shrink! automagically makes the contents fit on one slide
	\item \verb!allowframebrakes! splits contents of a frame if it does not fit
\end{itemize}
\item The \verb!\framesubtitle! creates a secondary slide title
\item The first slide is best created using the \verb! \begin{frame}[plain] \titlepage \end{frame}! commands.
\end{itemize}
\end{frame}

\subsection{Table of contents}

\begin{frame}
\frametitle{We are here now...}
	\tableofcontents[currentsection, hideallsubsections]
\end{frame}

\begin{frame}
\frametitle{...in fact, even here}
	\tableofcontents[currentsection, currentsubsection]
\end{frame}

\begin{frame}[fragile,shrink]
\frametitle{Table of contents}
\begin{itemize}
\item Create outlines using \verb!\tableofcontents[<<options>>]!
\item The possible options include:
\begin{itemize}
	\item \verb!currentsection! (all sections but current are greyed out)
        \item \verb!currentsubsection! (all subsections but current are greyed out)
 	\item \verb!hideallsubsections! (all subsections are hidden)
	\item \verb!hideothersubsections! (all subsections of sections other than the current are hidden)
  	\item \verb!pausesections! (shows the table of contents incrementally)
	\item \verb!pausesubsections! (finer increments than \verb!\pausesections!)
	\item \verb!sections={<2-3>}! (only sections 2 and 3 are displayed)
	\item \verb!sectionstyle=<<1>>/<<2>>! (define style of current section (\verb!<<1>>!), other sections (\verb!<<2>>!) using \verb!show!, \verb!shaded! and \verb!hide!, e.g. \verb!sectionstyle=shaded/show!)
• 	\item \verb!subsectionstyle=<1>/<2>/<3>! (define style for current subsection (\verb!<<1>>!), other subsections in current section (\verb!<<2>>!), subsections in other sections (\verb!<<3>>!))
\end{itemize}
\item The commands \verb!\section!, \verb!\subsection!, etc.\ make a structure for tables of contents (outlines are independent of slide titles)
\end{itemize}
\end{frame}

\subsection{Using boxes and images}

\begin{frame}[fragile]
\frametitle{Boxes}
\begin{itemize}
\item Use the \verb!\begin{<<env>>} ... \end{<<env>>}! command for predefined environments (e.g.\ \verb!definition!, \verb!theorem!, \verb!proof!, \verb!example!, \verb!corollary!, etc.) - not too pretty
\item Alternatively, use fancy boxes
\item Use the \verb!columns! environments for multiple columns
\end{itemize}
\vspace{-5ex}
\begin{columns}
\column{0.5\textwidth}
\begin{theorem}
	If\ $P$, then $Q$.
\end{theorem}
\setbeamercolor{uppercol}{fg=white,bg=red!54!black}
\setbeamercolor{lowercol}{fg=black,bg=red!10}
\begin{beamerboxesrounded}[upper=uppercol,lower=lowercol,shadow=true]{Theorem}
  If\ $P$, then $Q$.
\end{beamerboxesrounded}
\column{0.5\textwidth}
\begin{example}
	Consider $P=...$
\end{example}
\setbeamercolor{uppercol}{fg=white,bg=green!40!black}
\setbeamercolor{lowercol}{fg=black,bg=green!10}
\begin{beamerboxesrounded}[upper=uppercol,lower=lowercol,shadow=true]{Proof}
  Suppose that $P$ holds...
\end{beamerboxesrounded}
\end{columns}
\begin{columns}
\column{\textwidth}
\setbeamercolor{uppercol}{fg=white,bg=blue!40!black}
\setbeamercolor{lowercol}{fg=black,bg=blue!10}
\begin{beamerboxesrounded}[upper=uppercol,lower=lowercol,shadow=true]{Corollary}
  $Q$ holds
\end{beamerboxesrounded}
\end{columns}
\end{frame}

\begin{frame}[fragile]
\frametitle{Including images}
\begin{itemize}
\item Include images using the standard \verb!figure! environment
\item Beamer supports \verb!\includegraphics!, \verb!\pgfimage!, \verb!\pgfuseimage! and more
\end{itemize}
\begin{figure}
\includegraphics[height=100pt]{images/titlepage/ou}
\caption{Oxford University logo}
\end{figure}
\end{frame}


\section{Overlays}


\begin{frame}[fragile]
\frametitle{Simple overlays}
\begin{itemize}
\item<1-> Use \verb!\begin{itemize} \item<x-> \end{itemize}! to display bullet points incrementally
\item<2-> Alternatively, use the \verb!\pause! command, which displays contents of the slide up to the first marker, then up to the second marker, etc.
\end{itemize}
\pause
\begin{table}
\begin{tabular}{lcccc}
        & A & B & C & D \\\hline
  X     & 1 & 2 & 3 & 4 \pause\\
  Y     & 3 & 4 & 5 & 6 \pause\\
  Z     & 5 & 6 & 7 & 8
\end{tabular}
\end{table}
\end{frame}

\begin{frame}[fragile]
\frametitle{More complex overlays}
\begin{itemize}
\item The \verb!\uncover<x->! command orders the displaying of items.
\end{itemize}
\begin{semiverbatim}
\uncover<1->{\alert<0>{class helloWorld}}
\uncover<1->{\alert<0>{\{}}
\uncover<2->{\alert<2>{  public static void main(String args[])}}
\uncover<0->{\alert<0>{  \{}}
\uncover<3->{\alert<3>{    System.out.println("Hello World!");}}
\uncover<1->{\alert<0>{  \}}}
\uncover<1->{\alert<0>{\}}}
\end{semiverbatim}
\begin{itemize}
\item \verb!\alert<x>{<<text>>}! \alert<4->{colours text red on the $x$-th iteration of displaying items}
\end{itemize}
\end{frame}


\section{What else is possible}


\subsection{Making most of the Beamer class}

\begin{frame}[fragile]
\frametitle{What next}
\begin{itemize}
\item<1-> This presentation uses only a fraction of Beamer's capabilities
\item<1-> See the Beamer User Guide to learn how to: 
\begin{itemize}
\item<2-> create slide transitions
\item<3-> add notes
\item<4-> print handouts
\item<5-> add multimedia (sound, video)
\item<6-> ...and much more!
\end{itemize}
\item<7-> Alternatively, see \verb! http://www.matthiaspospiech.de/latex/vorlagen/! \verb! beamer/content/beamer-examples/!\\ for a shorter, example-based guide
\end{itemize}
\end{frame}
\end{document}
