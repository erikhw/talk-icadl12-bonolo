% Use this file as a template for your beamer slides.

\documentclass{beamer}

%\usetheme[secheader]{Madrid}
\usetheme{Madrid}

%\usepackage{beamerthemesplit}
\setbeamertemplate{items}[square] % Ma square bullets ni mbama zoona!
\usepackage{amsmath}
\usepackage{amsfonts}
\usepackage{amssymb}
\usepackage{amsthm}
\usepackage{graphicx}
\usepackage{setspace}
\usepackage{url}

% Footer symbols
\long\def\symbolfootnote[#1]#2{\begingroup%
\def\thefootnote{\fnsymbol{footnote}}\footnote[#1]{#2}\endgroup}

% Remove navigation symbols
\setbeamertemplate{navigation symbols}{}

\title[Title]{Bonolo\symbolfootnote[2, frame]{\tiny Sesotho word for easy.}}
\subtitle[Subtitle]{A General Digital Library System for File-based Collections}
\author[Author(s)]{{\bf Lighton Phiri} \and Kyle Williams \and Miles Robinson\\Stuart Hammar \and Hussein Suleman}
\institute[Institute]{Digital Libraries Laboratory\\Department of Computer Science\\University of Cape\\Cape Town, South Africa}
\date{November 13, 2012}


\usepackage{Sweave}
\begin{document}

% Remove footline for title slide/frame
%{
%\setbeamertemplate{footline}{} 
%\begin{frame}
%  \titlepage
%\end{frame}
%}
%\addtocounter{framenumber}{-1}

\maketitle

%first slide begins

\begin{frame}

\frametitle{First Slide...}

%contents of the first slides...

An itemized list:

\begin{itemize}

\item In beamer, a \emph{frame} is what you would normally call a ``slide''.

\item itemized list 2\footnote{This doesn't seem to be working.}

\item itemized item 3

\end{itemize}

A displayed formula:

\[
\sum_{k = 1}^{n} \binom{n}{k} x^{n - k} y^{k} = (x + y)^{n}
\]

\begin{theorem}[HSD]
Hardware\& software dependencies
\end{theorem}[HSD]

Remark: In addition to the \emph{theorem}  environment, beamer has the the following  predefined environments:
\emph{corollary}, \emph{definition}, \emph{example} and \emph{proof}.

\end{frame} %end of first slide


\begin{frame}[fragile] %second frame

\frametitle{Frames containing verbatim text...}

\begin{itemize}
\item \emph{Frames} containing verbatim text require special attention. Such frame must be marked as \emph{fragile}.
\item This is accomplished as follows

 \begin{verbatim}
 \begin{frame}[fragile]
 %...frame contents..
 \end{frame}
 \end{verbatim}

\item The \verb!\begin{frame}...\end{frame}! block may be repeated any number of times to produce a sequence of slides.
\item The \verb!\frametitle{...}! command puts a title on the slide. Although its use is optional, it is only in very rare cases where omitting a slide title would make sense.
\end{itemize}

\end{frame}

\begin{frame}[fragile]

\frametitle{Including Graphics in Frames...}

\begin{itemize}
\item Beamer recognizes images in any of the PDF, PNG and JPG formats. (Note that PostScript (eps format)  is not among these.)
\item It is important to know that beamer formats its output to a size of 5.04 in by 3.78 in.
\item Example: The code
\begin{verbatim}
\begin{center}
  \includegraphics[width=4in,height=2in]{icadl2012_bonolo-ci_evaluation_graph.png}
\end{center}
\end{verbatim}
results...
\end{itemize}
\end{frame}

\begin{frame}

\frametitle{UX - End User Interface}

\begin{center}
  %\includegraphics[width=.95\textwidth]{figure.png}
\includegraphics{icadl2012_bonolo-eval-ux-ui}
\end{center}

\end{frame}


\begin{frame}

\frametitle{UX - Curator Interface}

\begin{center}
  %\includegraphics[width=.95\textwidth]{figure.png}
\includegraphics{icadl2012_bonolo-eval-ux-ci}
\end{center}

\end{frame}

\begin{frame}

\frametitle{Performance - Directory Strucutre}
\begin{center}
  %\includegraphics[width=.95\textwidth]{figure.png}
\includegraphics{icadl2012_bonolo-eval-perf}
\end{center}
\end{frame}

\begin{frame}
\begin{center}
\includegraphics{icadl2012_bonolo-eval-ux-ci}
\end{center}
\end{frame}

\begin{frame}
\begin{center}
\includegraphics[width=0.9\textwidth, height=0.9\textheight]{hierarchical-structure.pdf}
\end{center}
\end{frame}

\end{document}
